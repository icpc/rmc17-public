\problemname{Particle Collision}

Particle colliders are difficult to build and experiments are costly
to run.  Before running any real experiments it is better to do a
simulation to test out the ideas first.  You are required to write a
very simple simulator for this problem.

There are only three particles in this system, and all particles are
confined to an infinite plane so that they can be modelled as circles.
Their locations are specified only by the $x_i$ and $y_i$ coordinates
of their centers ($1 \leq i \leq 3$).  All three particles have the
same radius $r$, and are initially stationary.

We are given a vector $(x_v, y_v)$ specifying the direction particle
$1$ will move when the experiment starts.  When particle $i$ hits
particle $j$, particle $j$ will start moving in the direction
perpendicular to the tangent at the point of the contact, away from
particle $i$.  Particle $i$ will cease to exist and be converted to
radiation.  A moving particle that does not hit another will continue
moving indefinitely.

There are a number of possible scenarios your simulator should identify:
\begin{enumerate}
\item particle $1$ hits particle $2$, which in turns hits particle $3$;
\item particle $1$ hits particle $3$, which in turns hits particle $2$;
\item particle $1$ hits particle $2$, which moves indefinitely;
\item particle $1$ hits particle $3$, which moves indefinitely;
\item particle $1$ moves indefinitely.
\end{enumerate}

\section*{Input} 

The input contains four lines.  The first three lines each contains 
two integers $x_i$ and $y_i$ ($|x_i|,|y_i| \leq 1000$), describing 
particles $1$, $2$, and $3$ in this order.  The fourth line contains three 
integers $x_v$, $y_v$, and $r$ ($|x_v|, |y_v| \leq 1000$, $0 < r \leq 50$).

You may assume that no two particles touch or overlap initially, and that
the distance between the centers of particles $2$ and $3$ 
is greater than $4r$.

\section*{Output}
Output a single integer giving the number ($1$--$5$) identifying the
scenarios described above.

Although you should take care of your calculations, it is guaranteed that the
outcome would not change if the initial vector $(x_v,y_v)$ is rotated by one
degree either way.
