\problemname{Heart Rate}

A common method for determining your own heart rate is to place your
index and third finger on your neck to the side of your windpipe.  You
then count how many beats you feel in a span of $15$ seconds, multiply
that number by four and that gives you a measure of your heart rate in
beats per minute (BPM).  This method gives a good estimate, but is not
quite accurate. In general, if you measure $b$ beats in $p$ seconds
the BPM is calculated as $\frac{60b}{p}$.

For this problem, we assume that all heart rates are constant and do
not change. If $t$ is the amount of time (in seconds) between each
beat of your heart, we define your Actual Beats Per Minute (ABPM) as
$\frac{60}{t}$.  
%Given the number of beats measured in an interval (as
%described above), determine the minimum and maximum of what the ABPM
%could be as well as what their measure would give.

\section*{Input}

The input starts with an integer $N$ ($1 \leq N \leq 1000$) indicating the 
number of cases to follow.  Each of the next $N$ lines specify one case, 
consisting of the integer $b$ ($2 \leq b \leq 1000$) as well as $p$ ($0 < 
p < 1000$) as described above.  The value of $p$ is a real number specified 
to 4 decimal places.


\section*{Output}

For each case, print on a single line the minimum possible ABPM, the
calculated BPM, and the maximum possible ABPM, separated by a space.
Your answer will be considered correct if its absolute or relative
error does not exceed $10^{-4}$.
