\problemname{Frosh Week}

Professor Zac is trying to finish a collection of tasks during the first week
at the start of the term.
He knows precisely how long each task will take, down to the millisecond.
Unfortunately, it is also
Frosh Week. Zac's office window has a clear view of the stage where loud music
is played. He cannot focus on any task when music is blaring.

The event organizers are also very precise.  They supply Zac with intervals
of time when music will not be playing. These intervals are specified by
their start and end times down to the millisecond.

Each task that Zac completes must be completed in one quiet interval.
He cannot pause working on a task when music plays (he loses his train of
thought). Interstingly, the lengths of the tasks and quiet intervals are such
that it is impossible to finish more than one task per quiet interval!

Given a list of times $t_i$ (in milliseconds) that each task will take
and a list of times $\ell_j$ (in milliseconds) specifying the lengths of the
intervals when no music is being played, what is the maximum number of tasks
that Zac can complete?

\section*{Input} The first line of input contains a pair of integers $n$ 
and $m$, where $n$ is the number of tasks and $m$ is the number of time 
intervals when no music is played. The second line consists of a list of 
integers $t_1, t_2, \ldots, t_n$ indicating the length of time of each 
task. The final line consists of a list of times $\ell_1, \ell_2, 
\ldots, \ell_m$ indicating the length of time of each quiet interval 
when Zac is at work this week.

You may assume that $1 \leq n,m \leq 200,000$ and $100,000 \leq t_i, \ell_j \leq 199,999$
for each task $i$ and each quiet interval $j$.

\section*{Output}
Output consists of a single line containing a single integer indicating the
number of tasks that Zac can accomplish from his list during this first week.
