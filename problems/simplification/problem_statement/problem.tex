\problemname{Polyline Simplification}

Mapping applications often represent the boundaries of countries,
cities, etc.\ as polylines, which are connected sequences of line
segments.  Since fine details have to be shown when the user zooms
into the map, these polylines often contain a very large number of
segments.  When the user zooms out, however, these fine details are
not important and it is wasteful to process and draw the polylines
with so many segments.  In this problem, we consider a particular
polyline simplification algorithm designed to approximate the original
polyline with a polyline with fewer segments.

A polyline with $n$ segments is described by $n+1$ points $p_0 = (x_0,
y_0), \ldots, p_n = (x_n, y_n)$, with the $i$th line segment being
$\overline{p_{i-1}p_i}$ .  The polyline can be simplified by removing
an interior point $p_i$ ($1 \leq i \leq n-1$), so that the line
segments $\overline{p_{i-1}p_i}$ and $\overline{p_ip_{i+1}}$ are
replaced by the line segment $\overline{p_{i-1}p_{i+1}}$.  To select
the point to be removed, we examine the area of the triangle formed by
$p_{i-1}$, $p_i$, and $p_{i+1}$ (the area is $0$ if the three points are
colinear), and choose the point $p_i$ such that the area of the
triangle is smallest.  Ties are broken by choosing the point with the
lowest index.  This can be applied again to the resulting polyline,
until the desired number $m$ of line segments is reached.

Consider the example below.
\begin{center}
\includegraphics{diagram.pdf}
\end{center}
The original polyline is shown at the top.  The area of the triangle
formed by $p_2$, $p_3$, and $p_4$ is considered (middle), and $p_3$ is
removed if the area is the smallest among all such triangles.  The
resulting polyline after $p_3$ is removed is shown at the bottom.

\section*{Input}

The first line of input contains two integers $n$ ($2 \leq n \leq
200\,000$) and $m$ ($1 \leq m < n$).  The next $n+1$ lines specify
$p_0, \ldots, p_n$.  Each point is given by its $x$ and $y$
coordinates which are integers between $-5000$ and $5000$ inclusive.
You may assume that the given points are strictly increasing in
lexicographical order.  That is, $x_i < x_{i+1}$, or $x_i = x_{i+1}$
and $y_i < y_{i+1}$ for all $0 \leq i < n$.


\section*{Output}

Print on the $k$th line the index of the point removed in the $k$th
step of the algorithm described above (use the index in the original
polyline).
